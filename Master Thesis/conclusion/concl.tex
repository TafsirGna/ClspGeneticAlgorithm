
Dans ce travail, nous avons présenté	les problèmes de dimensionnement de lots. Nous avons montré que la résolution de ces types de problèmes soulève certains défis intéressants. Nous nous sommes en particulier intéressés au PSP. Le \emph{Pigment Sequencing Problem} (PSP) fait partie de ces types de problèmes plus précisément du DLSP. Plusieurs méthodes de résolution du PSP ont été testées au cours de récentes recherches. Il s'agit des approches CP et SA. Cependant, aucune approche basée sur les algorithmes génétiques n'avait été proposée alors que ces derniers ont montré leur efficacité sur d'autres problèmes d'optimisation.\\
	\hspace*{.5cm} Nous avons proposé deux méthodes de résolution du PSP en l’occurrence les algorithmes génétiques parallèles hiérarchiques : \emph{Hierarchical Coarse-grained and Master-slave Parallel Genetic Algorithm} (HCM-PGA) et \emph{Hierarchical Fine-grained and Coarse-grained Parallel Genetic Algorithm}(HFC-PGA). L'approche HCM-PGA divise la population globale en sous-populations ou îlots auxquelles sont appliqués les opérations génétiques telles le croisement, la mutation ainsi que la sélection. Également, l'approche HCM-PGA confie à différents nœuds ou threads la tâche d'appliquer les opérations génétiques les plus gourmandes en ressource afin d'accélérer la recherche. Quant à l'approche HFC-PGA, elle divise dans la même logique la population globale en sous-populations; ces dernières étant disposées suivant une topologie de connexion particulière favorisant le chevauchement des bons chromosomes et la dissémination du matériel génétique.\\
	\hspace*{.5cm} Dans la troisième et dernière partie, nous avons validé nos deux propositions de méthodes de résolution à travers des tests que nous avons effectués sur deux groupes d'instances. Le premier est proposé dans la bibliothèque CSPlib et le second dans la bibliothèque Opthub. Ces tests ont permis d'analyser le comportement de nos deux propositions et de prouver qu'elles parviennent en moins raisonnable à trouver de solutions approchant la solution optimale.\\
	\hspace*{.5cm}Le travail présenté dans ce document est un bon point de départ pour de futures développements et extensions utilisant les algorithmes génétiques. Il serait par exemple intéressant de penser à des algorithmes de recherche locale qui améliorent significativement la qualité d'un chromosome tout en étant rapide sur les chromosomes des instances complexes testées dans notre étude ou encore penser à des moyens de croisement plus intelligents qui améliore la population. Et enfin, l'application des algorithmes génétiques parallèles et hiérarchiques à des problèmes encore plus complexes à plusieurs machines ou à plusieurs niveaux. 