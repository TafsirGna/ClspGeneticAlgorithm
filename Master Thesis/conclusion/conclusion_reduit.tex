\paragraph{}
Nous étions partis du constat que le système actuel de gestion de notes à l'Université d'Abomey-Calavi était perfectible.
Pour cela, nous avons décidé de mettre en place une plateforme qui sera fonctionnelle dans chaque entité de ladite université et qui résolvera les problèmes
liés à la gestion des notes passant par les voies que sont la gestion des offres de formations, la gestion des notes, la gestion des reprises et la gestion des jurys.
A la fin de ce travail, le système qu'on a produit (dans le cadre de l'IFRI) permet à l'administrateur du système d'insérer des filières, des cycles et toute une offre
de formation. Il lui permet également d'affecter des enseignants aux ECUE des offres de formations. Chaque enseignant sera notifié par mail et pourra donc se connecter
et voir la liste des ECUE à lui attribués ainsi que les informations relatives à cette dernière. Il pourra également voir la liste des étudiants préalablement inscrits
à ses ECUE et dont les comptes ont déjà été validés par l'administrateur. Il pourra donc leurs attribuer des notes. Enfin, les étudiants peuvent se connecter et voir leurs notes.
Toutefois, des perspectives intéressantes sont envisagées pour rendre le système plus complet. Voici en quelques points ces perspectives :
\begin{itemize}
 \item inclure à notre plateforme la gestion des reprises ;
 \item intégrer un module de validation des résultats par un jury avant leurs publications ;
 \item intégrer un module qui permet de faire l'analyse statistique des notes.
\end{itemize}

